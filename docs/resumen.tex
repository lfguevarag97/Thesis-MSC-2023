%%%%%%%%%%%%%%%%%%%%%%%%%%%%%
% RESUMEN <==>  ABSTRACT	%
%%%%%%%%%%%%%%%%%%%%%%%%%%%%%
%Resumen de la propuesta. (1/2 página). \\
Modelar dinámicamente un sistema es importante para representar de manera simplificada su comportamiento sin recurrir a la experimentación, que para sistemas complejos, puede representar un alto consumo de recursos. También es útil para predecir comportamientos a largo plazo; observar, analizar, e interpretar resultados, frente a diferentes entradas que afectan directa o indirectamente a el o los subsistemas presentes; establecer conclusiones y apoyar la toma de decisiones futuras.\\
% 

Este trabajo estudia el comportamiento productivo de ganado bovino y bufalino del Centro Agropecuario del Servicio Nacional de Aprendizaje (SENA), ubicado en la ciudad de Popayán (CA-SENA-POP), teniendo en consideración diferentes variables, entre ellas, la producción de leche, el consumo de agua, consumo de forrajes, consumo de concentrado, generación de desperdicios y finalmente utilidades por venta de producciones de leche y carne. El objetivo es estructurar y evaluar un modelo basado en ecuaciones diferenciales que facilite observar respuestas del sistema productor ganadero ante diferentes entradas, en pro de predecir comportamientos a largo plazo.\\

Este proyecto adapta de manera flexible la metodología CRISP-DM, con la que se busca ajustar conjuntos de datos a curvas lineales y no lineales, de tal forma que se proceda a modelar dinámicamente las respuestas del sistema productor ganadero frente a diferentes entradas. Este proyecto también utiliza conjuntos de datos debidamente registrados y proveídos por el SENA tanto de manera análoga como digital. Los resultados obtenidos pueden suponer un punto de referencia para prever situaciones futuras y evaluar el rendimiento del modelo frente a nuevos datos.\\

{\bf Palabras y Conceptos Clave}: Ajuste de curvas, Modelamiento Dinámico, Producción ganadera.