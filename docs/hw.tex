

El hardware se entiende como todo componente físico y tangible de un sistema de cómputo. En un proyecto  electrónico, el hardware abarca desde los circuitos eléctricos, la armadura del circuito y el conjunto de elementos, sensores, y actuadores que permiten ejecutar el software y las acciones lógicas necesarias para dar solución a una necesidad en particular.\\

\textit{\textbf{Observación: }En la escala de este proyecto, el hardware comprende en su mayoría la estructura física del prototipo, los sensores y los actuadores  manipulados por señales de control.}

\subsection{Actuadores}
\subsubsection{Actuadores Electrónicos}
\begin{itemize}
    \item \textbf{Servo}
    Los servomotores son motores similares a los de corriente directa con la posibilidad de ubicarse en una posición especifica dentro de su rango de operación. Están conformados por un motor, un mecanismo reductor y un circuito de control. Generalmente poseen potencia proporcional para cargas mecánicas; por lo tanto no consumen mucha energía a diferencia de otros ejemplares como los motores paso a paso.
\end{itemize}
\subsubsection{Actuadores Mecánicos}
\begin{itemize}
    \item \textbf{Mecanismo Biela-Manivela}
    Este mecanismo permite transformar movimiento circular en movimiento de traslación o viceversa. De manera esquemática, el mecanismo se crea con dos barras unidas por un par cinemático. Uno de los extremos de la barra que rota (la manivela) esta unida a un punto fijo siendo éste el centro de giro; su otro extremo está unido a la biela. El extremo restante de la biela se encuentra unido a un pistón que se mueve en línea recta.
    \item \textbf{Mecanismo de Tornillo sin Fin}
    Este mecanismo es descrito previamente en la sección \ref{endscrew}.
    
\end{itemize}
