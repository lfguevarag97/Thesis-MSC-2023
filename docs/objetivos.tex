%Objetivos 

%\subsection{Objetivo General}
\section{Objetivo General}
Estructurar y evaluar un modelo dinámico que simule el comportamiento de sistemas de producción ganadera en el Centro Agropecuario del Servicio Nacional de Aprendizaje, ubicado en el municipio de Popayán, en departamento del Cauca (CA-SENA-POP).

%\subsection{Objetivos Específicos 10 - 19}
\section{Objetivos Específicos}\label{objesp}

Los siguientes objetivos propuestos se ejecutarán a lo largo del desarrollo del este proyecto:

\begin{itemize} %(Verbos en Imperativo o Infinitivo, establecer, diseñar, estimar, etc)
    \item Identificar entradas, salidas, variables y parámetros asociados al sistema de producción ganadero objetivo.
    \item Identificar modelos dinámicos que permitan pronosticar la producción de un sistema ganadero objetivo.
    \item Implementar métodos de solución de los modelos dinámicos de producción ganadera seleccionados.
    \item Evaluar la estimación de los modelos seleccionados comparando la solución con datos empíricos recogidos.
\end{itemize}

\section{Delimitaciones y Alcances} \label{limites}
\begin{itemize}
	\item No se tendrán en consideración a animales que no participen en la producción del hato ganadero bovino y bufalino del CA-SENA-POP
% 	\item No se tendrán en consideración otras especies diferentes a los bovinos o bufalinos.
% 	\item Se está abierto a considerar la posibilidad de incluir un analisis que tenga en cuenta factores genéticos, más no esta incluido como objetivo del proyecto.
	\item El modelo se plantea con bovinos y bufalinos de producción lechera pertenecientes a distintos ciclos de lactancia, indiferentemente de su procedencia genética.
    \item Las codificaciones, algoritmos o ``scripts'' requeridos para analizar datos de la producción ganadera se realizan en software especializado como R, Matlab o Python.
	\item El ajuste de datos de producción lechera se realiza mediante métodos numéricos de interpolación lineales o no lineales según corresponda.
	\item El modelo se basa en datos existentes, verificables, análogos o digitales, y aprobados para su uso en este estudio, por parte de la producción ganadera del CA-SENA-POP.
	\item El modelo resultante podrá estimar comportamientos de variables de la producción ganadera y será evaluada usando datos existentes como referencia.
	\item Los datos suministrados por parte de CA-SENA-POP no serán considerados como entregables por motivos de acuerdos de confidencialidad.
% 	\item Tengo que confirmar la participacióin de los datos de Grupo Alpina sino no tengo datos y no podría solo dedicarme al Cauca sino donde me den datos para analizar (Pero de los casos,tengo miedo sr stark....).
\end{itemize}
\section{Entregables}

\begin{itemize}
	\item Documentación descriptiva, explicativa y argumentativa que evidencie el proceso de entendimiento, preparación, modelado, análisis, diseño, evaluación y desarrollo del proyecto.
	\item Modelo dinámico de referencia para estimación de producciones ganaderas en el CA-SENA-POP.
\end{itemize}
