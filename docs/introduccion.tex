%%%%%%%%%%%%%%%%%%%%
% INTRODUCCION
%%%%%%%%%%%%%%%%%%%%
%%Introducción general al problema que se va a resolver. (1 %%página)
%%%%\begin{align*}
%%%%\sin(x)=e^{2 \pi \theta \widehat{g}}\\
%%%%\sin(x)=e^{2 \pi}+\sin(x)
%%%%\end{align*}

En los últimos años, el sector agropecuario ha ido retomando gran importancia en Latinoamérica y el mundo, debido a múltiples factores entre los cuales se puede resaltar el crecimiento poblacional, que lleva consigo el incremento del consumo de alimentos. Esto representa una necesidad que puede y debe ser resuelta con el crecimiento de la producción agropecuaria de hasta un 70\% para satisfacer la demanda alimenticia a nivel mundial. Para ello es de vital importancia que los países en desarrollo como Colombia, entren a formar parte del mercado altamente productivo y competitivo haciendo uso de las fuentes que le aportan una ventaja comparativa en la región que radican, siendo estas la abundancia de recursos hidrícos y terrestres \cite{fao}.\\

Actualmente, Brasil es el segundo mayor proveedor de alimentos y productos agropecuarios a nivel mundial y cuenta con mejoras continuas en los procesos de producción, que respaldan el rápido crecimiento de las exportaciones. La investigación agropecuaria ha sido uno de los factores clave para el incremento de la productividad en la región durante las últimas décadas, con el fin de hacer seguimiento al registro, tratamiento, manipulación y analisis  los datos y el desempeño de los sistemas de investigación y desarrollo agropecuario mediante la evaluación de resultados. Estos datos son una herramienta indispensable cuando se trata de evaluar el aporte de la I+D agropecuaria y de manera más general, el crecimiento económico de la región y el país.\\

En América Latina y más precisamente en Colombia, sobresale la formación de 55 Comités de Investigación Agrícola elegidos Localmente (CIAL). Estos están constituidos por agricultores experimentados que administran y conducen investigación y desarrollo en beneficio de toda la comunidad \cite{ashby}; así pues se representa la alta participación colombiana en este ámbito de desarrollo. En Colombia también se han incorporado nuevas entidades que aportan al desarrollo investigativo, productivo y competitivo como la Corporación Colombiana de Investigación Agropecuaria (AGROSAVIA), que de mano con la expedición de la Ley 1876 de 2017 donde se crea el Sistema Nacional de Innovación Agropecuaria (SNIA), buscan integrar la ciencia, y el desarrollo tecnológico del sector junto con la formación y prestación del servicio de extensión agropecuaria. En este sentido, la ciencia permite lograr avances en términos productivos, de innovación y de competitividad \cite{minagricultura}.\\

En el Cauca, de acuerdo con el Instituto Geográfico Agustín Codazzi, los suelos del departamento presentan casi todos los pisos térmicos de variadas fertilidades y profundidades, y con diversas vocaciones para su uso \cite{igac}. Por otra parte, aunque comúnmente pueda considerarse un sector primitivo o de poca inversión tecnológica, el sector agropecuario cuenta con el mayor potencial tecnológico para el desarrollo sostenible y es uno de los más tecnificados en la región. Esto debido a la inclusión de nuevas tecnologías, no solo para la ejecución de procesos, sino también para la gestión y el manejo de las grandes cantidades de datos, impulsando así, el desarrollo competitivo del campesino e incrementando su participación en el mercado a nivel regional, nacional y mundial.

No obstante, la inclusión de las actuales tecnologías puede significar una gran inversión económica para pequeños y medianos productores y cultivadores. Los altos costos significan un impedimento a la participación competitiva en el mercado ganadero y de cultivos varios, con lo que el mejoramiento de pequeñas partes del sistema total puede manifestarse en mejoras significativas en el día a día de los nuevos emprendedores ganaderos. Estos costos se pueden taxonomizar en costos de producción, mantenimiento, transporte, control, toma, registro, seguimiento, manipulación y representación de datos; que diariamente aportan al consumo de recursos limitados como los económicos, personales y tiempo. Para ello se requiere de nuevos enfoques que permitan mejorar algunas de las condiciones de trabajo de los cultivadores agrícolas promoviendo la integración científica, innovadora y competitiva.\\

Con base en todo lo anterior, se considera indispensable la inclusión de tecnologías y metodologías de tratamiento, simulación y modelamiento computacional de los datos presentes en la producción ganadera.

%%%%%%%%%%%%%%%%%%%%%%
% DESCRIPCION DEL PROBLEMA
%%%%%%%%%%%%%%%%%%%%%%

\section{Planteamiento del Problema}

El sector ganadero abarca diferentes especies de ganado tales como los bovinos, los ovinos, bufalinos, entre otros. En Colombia se tiene una alta presencia de producción de estos tipos especialmente en el ganado bovino y ovino. De igual forma, estos tipos de ganado tienen diferentes modalidades, entre las que se pueden mencionar el ganado para crianza, para producción de leche y sus derivados, y ganadería de la carne o doble propósito. En Colombia y el mundo se presentan constantes inversiones en materia tecnológica y metodológica para mejorar la forma en cómo el ganado es productivo y puede participar de forma competitiva en un mercado de consumo de calidad \cite{fao}.\\

Aunque en el Cauca se cuente con vastas hectáreas de diferentes fertilidades y condiciones térmicas apropiadas para el cultivo y desarrollo de la ganadería \cite{igac}, aún se presentan falencias en materia de inversión tecnológica, económica e investigativa, en donde los movimientos migratorios de campesinos desplazados y los efectos colaterales del conflicto armado que se ha presentado en el país, son las principales causas de estas falencias \cite{fao}. El emprendimiento en el ámbito de la producción ganadera, especialmente para pequeños y medianos productores, presenta oportunidades significativas para la implementación de enfoques avanzados, como la ingeniería, el modelamiento matemático y la incorporación de metodologías de análisis, clasificación y manejo cuantitativo de datos, incluyendo la Inteligencia Artificial.\\

Sin embargo, la adopción de software sofisticado y modelos matemáticos puede plantear desafíos para los productores rurales, ya sea debido a la falta de capacitación técnica o a la complejidad de las herramientas de software utilizadas para el manejo, análisis y procesamiento de datos. Esto, a menudo, conduce a que las decisiones se tomen utilizando enfoques tradicionales en lugar de enfoques sistémicos y basados en datos.

% Sin embargo, el emprendimiento de los pequeños y medianos productores da paso a nuevas oportunidades de intervención por parte de la ingeniería, modelamiento matemático e inclusión de metodologías de análisis, clasificación, tratamiento y manejo cuantitativo de datos por medios como la Inteligencia Artificial. \\

% El manejo de software sofisticado, métodos o modelos matemáticos; pueden presuponer brechas para los productores campesinos por la falta de capacitación técnica, o por la complejidad de uso de los programas para el manejo, análisis y procesamiento de datos, lo que resulta en que la toma de decisiones dependa de un enfoque tradicional, mas no sistémico.\\

% A pesar de ello, en el último siglo, la investigación a nivel global ha posibilitado la estimación del comportamiento de variables asociadas a la producciones agropecuarias tales como la leche, la carne, y el consumo de insumos, a través de modelos matemáticos que varían en complejidades lineales o no lineales como se menciona

A lo largo del último siglo, la investigación a nivel global ha permitido la estimación y predicción del comportamiento de variables asociadas a la producción agropecuaria, como la producción de leche, carne y el consumo de insumos, a través de modelos matemáticos de diferentes complejidades, tanto lineales como no lineales, como se documenta en \cite{lechepaisa1}, \cite{caprinos}, \cite{colombiaherd}. Adicionalmente a esto, en las últimas décadas se han desarrollado múltiples investigaciones en las que, computacionalmente se modelan sistemas dinámicos de producción lechera tales como \cite{janaina1}, \cite{janaina2}, \cite{hdmwallace}, \cite{avila1}; Estos avances proporcionan herramientas valiosas para respaldar la toma de decisiones en los sistemas de producción ganadera \cite{gavira}.\\

% lo que presupone una herramienta de ayuda para apoyar la toma de decisiones en los sistemas de producción ganadera

% En la ganadería lechera, se maneja el término de curvas de lactancia, 
% \pagebreak
% la inclusión de las ciencias y la formación de entidades de apoyo al sector agrario pueden aportar significativamente al desempeño de los  cultivos ganaderos. En Colombia, se han realizado estudios investigativos con ganado bovino y caprino en diferentes regiones del país de las que se pueden mencionar
Como se menciona en \cite{minagricultura}, \cite{ashby} y \cite{fao}; la inclusión de la ciencia y el apoyo de instituciones en el sector agrario pueden tener un impacto significativo en el rendimiento de las operaciones ganaderas. En Colombia, se han llevado a cabo estudios de investigación con ganado bovino y caprino en diversas regiones del país, como se ejemplifica en \cite{lechepaisa1}, \cite{lechepaisa2}, \cite{lechepastusa}; Por lo tanto, se busca fomentar la investigación similar en la producción ganadera del departamento del Cauca, tomando como referencia la producción ganadera tecnificada del Centro Agropecuario (CA) del Servicio Nacional de Aprendizaje (SENA) en el municipio de Popayán.\\

Basándonos en lo anterior, se propone el desarrollo de un modelo dinámico de producción ganadera en el Centro Agropecuario (CA) del Servicio Nacional de Aprendizaje (SENA) en Popayán, con el propósito de estimar los consumos, las producciones y los rendimientos ganaderos. Este modelo se plantea como una herramienta esencial para respaldar la toma de decisiones futuras de los productores agropecuarios, permitiéndoles mejorar la gestión y la planificación de sus actividades ganaderas, lo que, a su vez, puede contribuir al crecimiento sostenible del sector.

% por lo que se busca posibilitar un desarrollo investigativo similar en producciones ganaderas del departamento del Cauca; tomando como referencia la producción ganadera tecnificada del Servicio Nacional de Aprendizaje (SENA) ubicada en el municipio de Popayán.\\

% Con base en lo anterior se plantea un modelamiento dinámico de producción ganadera en el Centro Agropecuario (CA) del Servicio Nacional de Aprendizaje (SENA) ubicado en el municipio de Popayán, con la finalidad de estimar consumos, producciones, y ganancias ganaderas; que sirvan de apoyo en la toma de decisiones futuras del productor.