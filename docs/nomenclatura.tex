% Nomenclatura

% \renewcommand{\nomname}{Nomenclatura}

\DeclareRobustCommand{\SkipTocEntry}[5]{}
\addtocontents{toc}{\SkipTocEntry}

% (Pendiente): Esta pagina  se plantea en caso que sea necesario dejar una lista de acrónimos, abreviaciones o notaciones que lo requieran. En caso que no se requiera se eliminara del documento final.
\makenomenclature\label{nomenclatura}

\renewcommand\nomgroup[1]{%
  \item[\bfseries
  \ifstrequal{#1}{S}{Símbolos}{%
  \ifstrequal{#1}{T}{Acrónimos}{}}%
]}

\renewcommand{\nompreamble}{El siguiente apartado describe los símbolos y acrónimos que serán usados más adelante en el contenido de este documento:\\}

% \section*{Símbolos}

\nomenclature[S]{$\alpha_{s}$}{Ingesta total de 
alimento. \nomunit{[\%]}}
\nomenclature[S]{$\alpha_{x}$}{Ingesta total de: forrajes (x=f), concentrado (x=c) o agua (x=a). \nomunit{[\%]}}
\nomenclature[S]{$\chi^{2}$ , $\chi^{2}_{red}$}{Factor ``chi'' cuadrado, Factor ``chi'' cuadrado reducido}
% \nomenclature[S]{$\chi^{2}_{red}$}{Factor ``chi'' cuadrado reducido}

% \section*{Acrónimos y abreviaturas}

% \nomenclature[T]{$AGROSAVIA$}{Corporación Colombiana de Investigación Agropecuaria.}
\nomenclature[T]{$CA$}{Centro Agropecuario.}
% \nomenclature[T]{$CIAL$}{Comités de Investigación Agrícola elegidos Localmente.}

% \nomenclature[T]{$ICA$}{Instituto Colombiano Agropecuario.}
\nomenclature[T]{$IGAC$}{Instituto Geográfico Agustín Codazzi.}
% \nomenclature[T]{$SNIA$}{Sistema Nacional de Innovación Agropecuaria.}
\nomenclature[T]{$SENA$}{Servicio Nacional de Aprendizaje.}
\nomenclature[T]{$PUJC$}{Pontificia Universidad Javeriana, Cali.}
% \nomenclature[T]{$MA$}{Método de Ajuste.}
\nomenclature[T]{$LMA$}{Levenberg-Marquardt.}
\nomenclature[T]{$FM$,$FU$}{Fit Múltiple, Fit Único.}
% \nomenclature[T]{$FU$}{Fit Único.}
\nomenclature[T]{$WP$}{Weight-Produced}
\nomenclature[T]{$ME$}{Efectos Mixtos (en inglés: Mixed Effects)}
\nomenclature[T]{$NL$}{No LineaL (en inglés: Non-Linear)}
\nomenclature[T]{$AIC$}{Criterio de Información de Akaike}
\nomenclature[T]{$BIC$}{Criterio de Información Bayesiano}
% \nomenclature[T]{$CRISP-DM$}{En inglés: Cross Industry Standard Platform for Data Minning}
\nomenclature[T]{$OE$}{Objetivo específico}
\nomenclature[T]{$EDO$}{Ecuaciones Diferenciales Ordinarias}


\printnomenclature[3cm]
