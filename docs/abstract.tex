%%%%%%%%%%%%%%%%%%%%%%%%%%%%%
% RESUMEN <==>  ABSTRACT	%
%%%%%%%%%%%%%%%%%%%%%%%%%%%%%
%Resumen de la propuesta. (1/2 página). \\

Modeling a system dynamically is important to represent its behaviour in a simplified way without resorting to experimentation, which for complex systems, it could represent a high resource consumption. It is also useful to predict long term responses; observe, analyze and interpret results against different inputs that directly or indirectly affect subsystem(s) present; draw conclusions and support future decision making.\\

This work studies the functioning of livestock production in the Agricultural Center of the National Apprenticeship Service, Popayan sectional (CA-SENA-POP); taking into account a variety of variables such as milk yield; water, fodder and concentrated food consumption; waste generation and last but not least, profit from milk and meat production sales. The objective is to structure and assess a model based on differential equations that facilitates the visualization of the CA-SENA-POP livestock system against different situations for long term behaviour prediction.\\

This project adapts, in a felixible way, the CRISP-DM methodology, with which it is sought to adjust
data sets to linear and nonlinear curves in such a way that the responses of the livestock production system to different inputs are dynamically modeled. This project also uses properly registered and organized databases provided by the SENA institution,both analog and digital.
The results obtained are necessary to foresee future situations and evaluate the performance of the model against new data.\\


{\bf Keywords}: Curve fitting, Dynamic modeling, Livestock production.
