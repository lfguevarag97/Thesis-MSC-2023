%%Desarrollo de las guías de Arduino para el curso de BioFisica I y II.
%%
%%El objetivo de esta actividad es permitir expresar mediante una guía didactica, el manejo de Arduino para cualquier persona que tenga poca experiencia o nula, en el manejo de estos dispositivos.
%%
%%Las guías constarán de ejemplos básicos que pueden ser desarrollados con base en las monitorias dictadas por el monitor Luis Felipe Guevara Gómez y el profesor del curso Oscar Ramirez G.

Algun\@s de la actividades propuestas son las siguientes.
%%
%%\begin{itemize}
%%\item Manejo de servo motores.
%%\item RFID.
%%\item Infrarojo.
%%\item Pantalla display o 7 segmentos.
%%\item Medición de distancia.
%%\item Detección de movimiento.
%%\item Nivel de liquido o solido.
%%\item Sensor de Lluvia
%%\item Humedad de tierra
%%
%%\end{itemize}
%%
%%Cada guía debe estar en la capacidad de facilitar el uso de un sensor/modulo de Arduino, a aquellos que no tengan experiencia o estén comenzando con el uso de este.
%%Todas las guías deben tener como mínimo lo siguiente:
%%
%%\item Breve introducción del dispositivo a usar.
\item Descripción del aplicativo o el ejemplo a explicar.
%%\item Materiales.
%%\item Librerias (Descarga e instalación, Si lo requiere).
%%\item Otras aplicaciones del dispositivo explicado.
\item Un diagrama de conexión detallado que garantice su funcionamiento.(Paint, Fritzing, etc).
%%\item Código funcional explicado.
\item Imágenes o ayudas gráficas (Fotos, diagramas, tablas, screenshots, etc).
\item Prueba de su funcionalidad.
%%\item Bibliografía.
%%
%%
