\section{TAREAS}
\begin{itemize}
    \item  \textbf{Hacer PPT para el personal del SENA}
    \item  \textbf{Hacer resumen del trabajo para articulo cientifico}, formato genérico por el momento
    \item  \textbf{Hacer 
    formulario del trabajo (6 preguntas)}. Hablar  del modelado, del tabajo realizado y la precision del modelo. Posibilidad y trabajos futuros.
    \item Corregir preguntas formulario
    \item Cuadrar reunion (1 semana de anticipación)

\end{itemize}


\section{Ajuste de curvas}

Hablar de las tablas de los parámetros para cada tipo de ajuste de curva no lineal y cual permite una aproximación mas cercana a la producción neta de leche y por ende de ganancia por venta de esa leche


\section{Ajuste por mínimos cuadrados}
Hablar de cuales vacas se ajustan al modelo
Hablar de cuales vacas NO se ajustan al modelo
Hablar de los errores relativos
Mostrar los valores de k1, k2, k3 y k4
Mostrar las gráficas con los ajustes logrados y NO logrados\\

\textit{\textbf{Model.Minimize y la tabla de ajustes de k1, k2 , k3, k4 para cada vaca, errores relativos menores al 30\% etc}}

\subsection{Visualización de resultados}
Comparar los modelos ajustados para las vacas que presentan buen ajuste, mostrar los valores de k1, k2, k3, y k4... Con los modelos en matlab usando esos valores de k1, k2, k3, y k4.\\

Comparar las graficas de matlab de las 4 vacas que tienen buen ajuste (Python vs MATLAB)

\section{Funciones objetivo de la programación lineal}

Hablar de como se puede maximizar las ganancias (\$\$\$) a lea vez que se minimizan las emisiones de $CO_{2}$


