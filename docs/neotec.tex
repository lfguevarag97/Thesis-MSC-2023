% 
% 
% 
\begin{itemize}

\item \textbf{Arduino}

Es una plataforma electrónica de código abierto basado en software y hardware de fácil uso. Las placas de Arduino están en la capacidad de leer entradas, manipular sensores y transmitir información de manera remota, entre otros. Estas placas son reprogramables mediante una serie de instrucciones al microcontrolador y se pueden programar con múltiples lenguajes de programación \cite{arduinodef}.

\item \textbf{PLC}

Es una computadora lógica y programable utilizada para automatizar procesos electromecánicos, electro-neumáticos y electro-hidráulicos especialmente en líneas de montaje o procesos de producción\cite{plcref} .

\item \textbf{Raspberry Pi}

Son microordenadores de placa reducida de bajo costo que tiene como objetivo estimular la enseñanza de la informática aunque tiene usos incluso profesionales como en la robótica. A este dispositivo se le pueden agregar periféricos al igual que a un ordenador de escritorio y puede ser usado  como tal. Sus usos abarcan desde sistemas de cómputo básico, análisis de datos, representación de imágenes y vídeo, y aplicaciones básicas de automatización \cite{piref}.

%CARACTERÍSTICAS Y PROPIEDADES DE CADA UNO

%TABLA COMPARATIVA

\end{itemize}
