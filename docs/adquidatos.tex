% 
% 
% 

Como se mencionó en la Sección \ref{subsectionembebidos}, los sistemas electrónicos están en la capacidad de reaccionar a los estímulos del entorno. Esto hace referencia a los datos y señales de entrada provenientes de elementos electrónicos que suministran tal información. Para medir los datos que luego serán procesados e interpretados en acciones de la unidad central de procesos de un sistema computacional es necesario hacer uso de estos elementos:

\subsection{Sensores}

Son dispositivos que pueden variar sus propiedades ante magnitudes físicas o químicas del medio. Esto significa que pueden detectar su presencia y también su medida. A las variables medidas se les denomina variables de instrumentación y al transformarlas mediante el uso de un transductor se convierten en variables eléctricas. Un transductor es un elemento que recibe variaciones de energía y están en la capacidad de representarlas mediante otras variaciones de energía de salida \cite{instrum}. A diferencia de los transductores, los sensores están en contacto constante con la variable de instrumentación a la que se encuentran expuestos. Con base en esto se puede establecer que los sensores ``traducen''  o adaptan una señal de entrada para que otros dispositivos puedan interpretarla.\\

%%% Tek

Las señales de entrada son definidas como una manifestación abstracta, física o simbólica de información como por ejemplo las señales de audio, imágenes, manifestaciones climatológicas y del medio, físicas y químicas. Como los sensores entregan una variación de energía en la salida que a su vez representa la conversión de una señal de entrada de una variable de instrumentación en particular, esta debe ser modificada para que pueda ser interpretada correctamente por otros dispositivos de análisis de datos. Para ello es posible que sea necesario hacer uso de un acondicionamiento de la señal para representar la variable mediante una diferencia de potencial y dependiendo del tipo de sensor, es probable que se deba incluir una etapa de conversión de señal analógica o digital. \cite{instrum}.

\subsubsection{Clasificación}

Para determinar si un sensor requiere de una etapa de conversión ADC, los sensores son clasificados por sus características, que pueden ser del tipo de señal analógica o digital.

\begin{itemize}
    \item \textbf{Sensores de señales analógicas}
    Los sensores analógicos miden señales que están definidas para todos los valores de tiempo en un intervalo continuo. Esto quiere decir que para todo instante de tiempo existirá una señal de salida proporcional a un valor de magnitud de la variable medida \cite{benitez}.
    
    \item \textbf{Sensores de señales digitales}
    Los sensores digitales miden señales que están definidas por valores discretos, es decir que representan una señal mediante una variación discontinua en el tiempo y solo pueden tener un número finito de valores discretos \cite{benitez}.
\end{itemize}


\subsubsection{Características técnicas}
\begin{itemize}
    \item \textbf{Linealidad}
    Se dice que un sensor es lineal si su curva de calibración se asemeja a una línea recta. Esto quiere decir que su comportamiento de medida es lineal entre su variable de medida y la señal de salida que entrega el sensor.
    
    \item \textbf{Precisión}
    Es la capacidad de un instrumento de medición de entregar valores aproximadamente iguales a medida que se repiten los eventos de medición bajo las mismas condiciones.
    
    \item \textbf{Error}
    Es la discrepancia existente entre el valor medido (valor real) y el valor teórico de medición (valor ideal o de referencia).
    
    \item \textbf{Exactitud}
    Es la capacidad de un instrumento de medición de entregar valores que se aproximen al valor verdadero de la magnitud de medida.
    
\end{itemize}

% \subsubsection{Clasificación}
% ==================================================================
% ==================================================================
% ==================================================================
%  **********YO CREO QUE ESTA PARTE DE SOFTWARE VA MEJOR EN EL TRATAMIENTO DE LOS DATOS************************************.
% \subsection{Software}

% Se entiende como Software a todo soporte lógico e intangible necesario que hacen posible la realización de tareas especificas. Según el estándar 729 de la IEEE, se entiende como  el conjunto de programas, procedimientos, reglas, documentación y datos asociados que forman parte de las operaciones de un sistema de computo. Entre los componentes lógicos se puede hacer mención de de aplicaciones informáticas, el sistema operativo que permite que los programas funcionen correctamente y los programas de interfaz de usuario. \\

% Por lo general, el software se escribe en lenguajes de programación de alto nivel, es decir que son herramientas de programación fáciles y eficientes para los programadores debido a su semejanza al lenguaje natural, por su parte los sistemas de computo ejecutan estos programas en lenguaje maquina. La traducción entre estos lenguajes se realiza mediante un compilador que se encarga de ``traducir'' las instrucciones.

% \subsubsection{Clasificación}
% \begin{itemize}

%     \item \textbf{Sistema: }
%     El es tipo de software de alto nivel que separa las acciones del usuario en comparación de las acciones de bajo nivel que debe ejecutar el sistema de computo con respecto a su memoria, procesador, etc.
    
%     \item \textbf{Programación: }
%   Son las herramientas que le facilitan al programador programar los algoritmos informáticos mediante lenguajes de programación de manera aplicada. A groso modo poseen Editores de texto, compiladores, interpretes y entornos de desarrollo que cuentan con interfaces gráficas para el usuario.
    
% \end{itemize}

% % \subsubsection{Características}
% % \begin{itemize}
% %     \item \textbf{Programación o Codificación}
% % % \end{itemize}
% % 



% \subsubsection{Codificación}
% El proceso para  diseñar software se logra con el objetivo de idear la forma en que las acciones se realizarán en el marco de un proyecto para solventar una necesidad en particular. Hoy en día se considera programación al proceso de escritura de algoritmos computacionales mediante un editor de texto o herramienta integrada como los lenguajes de programación entre los que se encuentran Python, C y C++.\\

% Los algoritmos son un conjunto de instrucciones o reglas predefinidas de manera ordenada que permiten llevar a cabo una actividad mediante pasos consecutivos secuenciales o paralelos de manera no ambigua para poder realizar una actividad \cite{algoritmo}. Los algoritmos computacionales son algoritmos más sofisticados y precisos que permiten aprovechar las nuevas tecnologías y que al depender de una memoria limitada deben ser lo mas optimizados posible para que puedan procesar grandes cantidades de datos con capacidades finitas y generalmente a bajo costo. 

% ****************************************************
